\documentclass[a4paper,10pt]{article}
\usepackage[utf8]{inputenc}
\usepackage[pdftex]{graphicx}
\usepackage{amsmath}
\usepackage{listings}

%opening
\title{MAP2210 Aplicações em Álgebra Linear \\ ---Questão 1 P2---}
\author{Vitor Gonçalves Ribeiro Nº Usp:9379548}

\begin{document}

\maketitle

\begin{abstract} 
   Neste trabalho iremos aproximar numericamente uma equação diferencial de segunda ordem usando uma matriz esparsa, e analisaremos a ordem dessa aproximação, discutiremos o limite de erro da resolução computacional e abordaremos as vantagens em tempo na utilização de um modelo de matriz esparsa contra uma matriz normal.
\end{abstract}

\section{Modelo da Solução}
  Neste modelo iremos aproximar numericamente a equação diferencial de segunda ordem
  \begin{eqnarray}
  	u''(x) = f(x),\ x\ \epsilon\ (a,b)
  \end{eqnarray}
  Para isso discretizaremos a solução usando o Método das Diferenças Finitas, com esquemas diferentes entre as linhas da matriz, esses esquemas serão:
  \begin{eqnarray}
  	-\frac{1}{h^2}(u_{k-1} -2u_k + u_{k+1} ) = f_k
  \end{eqnarray}
  \begin{eqnarray}
    -\frac{1}{h^2}(2u_{k} -5u_{k+1} + 4u_{k+2} + u_{k+3} ) = f_k
  \end{eqnarray}
  \begin{eqnarray}
    -\frac{1}{12h^2}(-u_{k-2} +16u_{k-1} -30u_k + 16u_{k+1} - u_{k+2}) = f_k
  \end{eqnarray}
  Nos esquemas acima temos \textit{$u_k$ $\approx$ $u(x_k)$} e \textit{$f(x_k)$
  = $f_k$}, onde o conjunto de
  pontos \textit{$x_k$ = a + hk, k = 0, 1, . . . , n} discretiza o domínio de definição da solução.
  Note que o passo \textit{h = (b - a)/n} é determinado pela escolha de n, que será nossa variável de análise.\\
  Usaremos a equação (3) para a primeira linha da matriz (não temos um $u_{k-2}$ para essa linha), e a condição (2) para a ultima (não temos um $u_{k+2}$ nessa linha), para as demais utilizaremos a condição (4).\\
  Utilizando condições de contorno de Dirichlet, ou seja os valores de \textit{u(a)} e \textit{u(b)} são dados, podemos transformar nosso problema na solução da expressão:\\
  \setcounter{MaxMatrixCols}{20}
  {\tiny{
  \[
  \begin{bmatrix}
  	 2 & -5 & 4 & -1  & 0 & 0 & 0 & 0 & \dots & 0 & 0 & 0 & 0 & 0\\
  	 16 & -30 & 16 & -1  & 0 & 0 & 0 & 0 & \dots & 0 & 0 & 0 & 0 & 0\\
  	 -1 & 16 & -30 & 16  & -1 & 0 & 0 & 0 & \dots & 0 & 0 & 0 & 0 & 0\\
  	 0 & -1 & 16 & -30  & 16 & -1 & 0 & 0 & \dots & 0 & 0 & 0 & 0 & 0 \\
  	 0 & 0  & -1 & 16  & -30 & 16 & -1 & 0 & \dots & 0 & 0 & 0 & 0 & 0 \\
  	 \vdots & \vdots & \vdots &\vdots &\vdots & \vdots & \vdots & \vdots & \ddots & \vdots & \vdots & \vdots & \vdots & \vdots \\
  	 0 &  0 & 0 & 0  & 0 & 0 & 0 & 0 &\dots & -1 & 16 & -30 & 16 & -1 \\
  	 0 &  0 & 0 & 0  & 0 & 0 & 0 & 0 &\dots & 0 & 0 & 0 & 1 & -2 \\
  \end{bmatrix}
  *
  \begin{bmatrix}
    u_{1}\\
    u_{2}\\
    u_{3}\\
    u_{4}\\
    u_{5}\\
    \vdots\\
    u_{n-2}\\
    u_{n-1}\\
  \end{bmatrix}
  =
  \begin{bmatrix}
  	-a\\
  	-12a -u_0\\
  	-12a\\
  	-12a\\
  	-12a\\
  	\vdots\\
  	-12a + u_n\\
  	-a - u_n\\
  \end{bmatrix}
  \]
  }
}\\
Como \textit{$u_0$ = $u(a)$} e \textit{$u_n$ = $u(b)$} e são dados, elas vão para o lado da conhecido da equação.
\section{Modelo Computacional}
Para criar o modelo computacional descrito acima, foi criada a função P2Q1, onde recebe como parâmetros:
\begin{itemize}
\item f0: valor de $u_0$
\item f1: valor de $u_n$
\item x0: valor de $a$
\item x1: valor de $b$
\item n:  numero de divisões $n$
\item b:  vetor resultado $f_k$
\end{itemize}
Essa função cria a Matriz A, conforme descrito na sessão de Modelo da Solução usando a biblioteca do python scipy.sparse, para aproveitar da esparsidade da matriz, reduzindo o tempo de execução. A função é mostrada abaixo.\\
\begin{lstlisting}
def P2Q1(f0, f1, x0, x1, n, b):
    t = time()
    a = b*((x1-x0)/n)**2
    if(n>=5):
        A = lil_matrix((n-1,n-1))
        for i in range(n-1):
        if(i==0):
			A[i,i] = 2
		A[i,i+1] = -5
		A[i,i+2] = 4
		A[i,i+3] = -1
		a[i] = -a[i]
		elif(i==n-2):
			A[i,i] = -2
			A[i,i-1] = 1
			a[i] = -a[i] - f1
		else:
			a[i] = -12*a[i]
			if(i>1):
				A[i,i-2] = -1
			else:
				a[i] = a[i] + f0
			A[i,i-1]=16
			A[i,i]=-30
			A[i,i+1]=16
			if(i<n-3):
				A[i,i+2]=-1
			else:
				a[i] = a[i] + f1
	else:
		print("numero de colunas insuficientes")
	A = A.tocsr()
	return linalg_sparse.spsolve(A,a),time()-t
\end{lstlisting}
Ao final essa função retorna a resolução da equação, ou seja os valores de $u_k$, e o tempo de execução, esses resultados serão mostrados em tabelas na próxima sessão.
\section{Resultados}
Para obter os resultados desse problema foi usada a equação
\begin{eqnarray}
	-u''(x) = ( cos(x) - sin^2(x))exp(cos(x)) = f(x)
\end{eqnarray}
que a solução analítica é conhecida
\begin{eqnarray}
	u(x) = exp(cos(x))
\end{eqnarray}
dessa forma podemos calcular o erro, e assim aproximar a ordem $p$ do erro ($O(h^p)$)
\begin{table}[!ht]
\centering
\begin{tabular}{rccc}\hline\hline\\
 $n$ &$h_n=\displaystyle \frac{(T-t_0)}{n}$ & $\left|e(T, h_n)\right|_2$  & ordem $p_2$ \\ 
	\hline\hline
	\\
	128 & 4.90874e-02 & 2.65068e-04& --- \\
	256 & 2.45437e-02 & 1.71826e-05&3.94735e+00 \\
	512 & 1.22718e-02 & 1.08473e-06&3.98554e+00\\
	1024 & 6.13592e-03 & 6.79975e-08&3.99571e+00\\
	2048 & 3.06796e-03 & 4.24739e-09&4.00083e+00\\
	4096 & 1.53398e-03 & 2.31112e-10&4.19992e+00\\
	8192 & 7.66990e-04 & 2.00766e-10&2.03076e-01\\
	16384 & 3.83495e-04 & 9.31555e-10&-2.21413e+00\\
	32768 & 1.91748e-04 & 3.60026e-09&-1.95039e+00\\
	\\
  		\hline\hline
\end{tabular}
\caption{Método de Euler aplicado ao Problema de Cauchy em $t=T$. 
}
\label{tabmodel-1}
\end{table}

\begin{table}[!ht]
	\centering
	\begin{tabular}{rccc}\hline\hline\\
		$n$ &$h_n=\displaystyle \frac{(T-t_0)}{n}$ & $\left|e(T, h_n)\right|_\infty$  & ordem $p_\infty$ \\ 
		\hline\hline
		\\
		128 & 4.90874e-02 & 4.54881e-04&nan\\
		256 & 2.45437e-02 & 2.96326e-05&3.94023e+00\\
		512 & 1.22718e-02 & 1.87824e-06&3.97973e+00\\
		1024 & 6.13592e-03 & 1.17977e-07&3.99280e+00\\
		2048 & 3.06796e-03 & 7.38815e-09&3.99715e+00\\
		4096 & 1.53398e-03 & 4.61880e-10&3.99962e+00\\
		8192 & 7.66990e-04 & 2.58854e-10&8.35379e-01\\
		16384 & 3.83495e-04 & 1.18161e-09&-2.19054e+00\\
		32768 & 1.91748e-04 & 4.61141e-09&-1.96445e+00\\
		\\
		\hline\hline
	\end{tabular}
	\caption{Método de Euler aplicado ao Problema de Cauchy em $t=T$. 
	}
	\label{tabmodel-2}
\end{table}

\begin{table}[!ht]
	\centering
	\begin{tabular}{rccc}\hline\hline\\
		$n$ &$h_n=\displaystyle \frac{(T-t_0)}{n}$ & $tempo\ A\ sparsa(s)$  & $tempo\ A\ normal (s)$ \\ 
		\hline\hline
		\\
128 & 4.90874e-02 & 7.48713e-02 & 2.58684e-01\\
256 & 2.45437e-02 & 9.59039e-03 & 1.16580e-02\\
512 & 1.22718e-02 & 2.22425e-02 & 2.84448e-02\\
1024 & 6.13592e-03 & 3.70302e-02 & 1.16141e-01\\
2048 & 3.06796e-03 & 3.46773e-02 & 4.44659e-01\\
4096 & 1.53398e-03 & 9.52613e-02 & 2.06000e+00\\
8192 & 7.66990e-04 & 1.19611e-01 & 1.21814e+01\\
16384 & 3.83495e-04 & 2.02301e-01 & 9.48454e+01\\
32768 & 1.91748e-04 & 4.00943e-01 & Morto\\
\\
\hline\hline
\end{tabular}
\caption{Método de Euler aplicado ao Problema de Cauchy em $t=T$. 
}
\label{tabmodel-3}
\end{table}

\section{análise}


\end{document}