\documentclass[]{article}

%opening
\title{EP1 MAP2212}
\author{Vitor Gon\c{c}alves Ribeiro N$^\circ$ USP: 9379548}

\begin{document}

\maketitle

\section{Introdu\c{c}\~{a}o}
   Neste EP, o c\'{o}digo dever\'{a} estimar o valor de $\pi$ usando Monte Carlo, obtendo um erro menor que 0,05$\%$*$\pi$, estimando o tamanho da amostra "n"
\section{Desenvolvimento}
   Para o problema temos uma soma de binomiais onde o calculo do n seria 
   \begin{equation} 
	n = \frac{Z_{\gamma}^2 \sigma^2}{\epsilon^2}
   \end{equation}
   Come\c{c}o com uma estimativa pessimista do n com $\frac{\pi}{4} = \frac{1}{2}$, um intervalo de confian\c{c}a de 95$\%$ dando um $Z_{\gamma}$ = 1,960 com isso obtemos um n = 1.536.640.000
   Como estamos com uma estimativa pessimista precisamos recalcular o n depois de algumas interações, como queremos um erro de 0,05$\%$ aproveito para fazer 100 interações do Monte Carlo de para um conjunto de 0,05$\%$ do n, com isso já tendo uma primeira estimativa de $\bar{\pi}$, e com isso recalculando pi recursivamente até o número de interações superar n.
\section{Resultados e Discuss\~{a}o}
   esse c\'{o}digo gera um n da ordem de 4.200.000, e um $\pi$ da ordem de 3.141, que nos d\'{a} um erro em na ordem de 0.03$\%$
\section{Conclus\~{a}o}
    Usando uma estimativa pessimista podemos saber uma ordem inicial de interações do Monte Carlo, para aí assim podermos usar a o valor estimado do Monte Carlo para recalcular os dados e obter valores mais assertivos.
\section{Bibliografia}
   https://edisciplinas.usp.br/pluginfile.php/6827046/mod$\_$resource/content/3/Livro$\_$Stern.pdf
\end{document}
